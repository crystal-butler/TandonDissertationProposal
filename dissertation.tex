\documentclass[letterpaper]{tandon_thesis}
% \usepackage[medium]{titlesec}
\usepackage{titlesec}
\usepackage{calc}
\usepackage[subfigure]{ccaption}
\usepackage{subfigure}
\usepackage{tabularx}
\usepackage{algorithm2e}
\usepackage{url, graphicx, amsmath, amsfonts, amsthm}
\usepackage{url}
\usepackage{multirow}
\usepackage{multicol}
\usepackage{pifont}
\usepackage{color}
\usepackage{soul}
\usepackage[backend=biber,style=ieee,sorting=nyt]{biblatex}

\addbibresource{sample.bib}

%   Credit for the following environment to @lockstep: https://tex.stackexchange.com/questions/10104/two-bibliographies-with-two-different-styles-in-the-same-document
%   Print publications on the vita page without numbers.
\defbibenvironment{nolabelbib}
    {\list
        {}
        {\setlength{\leftmargin}{\bibhang}%
         \setlength{\itemindent}{-\leftmargin}%
         \setlength{\itemsep}{\bibitemsep}%
         \setlength{\parsep}{\bibparsep}}}
    {\endlist}
    {\item}

\titleformat{\chapter}[display]
{\bfseries\Huge}{}{0pt}{\LARGE}

\newtheorem{condition}{Condition}[section]
\newtheorem{definition}{Definition}[section]
\newtheorem{theorem}{Theorem}[section]
\newtheorem{lemma}[theorem]{Lemma}
\newtheorem{cor}[theorem]{Corollary}
\newtheorem{corollary}[theorem]{Corollary}
\newtheorem{claim}[theorem]{Claim}
\newtheorem{fact}[theorem]{Fact}
\newtheorem{example}[section]{Example}
\newtheorem{note}{Note}[section]

%% Some trickery to be able to add stuff in before the first chapter.
\def\nonumchapter#1{%
    \chapter*{#1}
    \addcontentsline{toc}{chapter}{#1}}

\def\prefacesection#1{%
    \chapter*{#1}
    \addcontentsline{toc}{chapter}{#1}}

\def\afterpreface{\newpage
    \pagenumbering{arabic}
    \typeout{Dalthesis preface pages completed.}
    }
    

    
%%%%%%%%%%%%%%%%%% for pangolin %%%%%%%%%%%%%

\newcommand{\techreport}[1]{}
\newcommand{\submission}[1]{#1}
\newcommand{\bda}	{\begin{eqnarray*}}
\newcommand{\eda}	{\end{eqnarray*}}
\newcommand{\bdalign}	{\begin{align*}}
\newcommand{\edalign}	{\end{align*}}
\newcommand{\define}	{\equiv}
\DeclareMathOperator*{\argmin}{arg\,min}


\newcommand{\be}{\begin{equation}}
\newcommand{\ee}{\end{equation}}
\newcommand {\beq}{\begin{equation}}
\newcommand {\eeq}{\end{equation}}
\newcommand {\bear}{\begin{eqnarray}}
\newcommand {\eear}{\end{eqnarray}}
\newcommand {\bearn}{\begin{eqnarray*}}
\newcommand {\eearn}{\end{eqnarray*}}
\newcommand {\barr}{\begin{array}}
\newcommand {\earr}{\end{array}}
\newcommand {\done} {\hfill\quad\vrule height4pt WIDTH4PT}
\def\cee#1#2{\left(\barr{c} #1\\ #2 \earr \right)}
\def\tA{\tilde{A}}
\def\tL{\tilde{L}}
\def\tM{\tilde{M}}
\def\real{{\rm I\!R}}
\def\cS{{\cal S}}
\def\cF{{\cal F}}
\def\hT{\hat{T}}
\def\naive{na\"{\i}ve}


\newcommand{\myvec}[1]{\mathbf{#1}}
\newcommand{\mymatrix}[1]{\mathbf #1}
\newcommand{\rank}{\mathrm{rank}}
\newcommand{\myspan}{\mathrm{span}}
\newcommand{\mynull}{\mathrm{null}}
\newcommand{\mysize}{\mathrm{size}}
\newcommand{\myenqueue}{\mathrm{enqueue}}
\newcommand{\mypop}{\mathrm{dequeue}}
\newcommand{\mymatching}{\mathrm{matching}}

%%%%%%%%%%%%%%%%%% for pangolin end %%%%%%%%%%%%%
    
\usepackage[babel=once,english=american,autostyle=tryonce,strict=true]{csquotes}


\usepackage[final]{hyperref}%?%hyperfootnotes=false
\hypersetup{%bookmarks=false,        % show bookmarks bar?
    unicode=true,           % non-Latin characters in Acrobat’s bookmarks
    pdftoolbar=true,        % show Acrobat’s toolbar?
    pdfmenubar=true,        % show Acrobat’s menu?
    pdffitwindow=false,     % window fit to page when opened
    pdfstartview={FitH},    % fits the width of the page to the window
    pdftitle={PhD Thesis},
    pdfauthor={Author Name},     % author name goes here
    pdfsubject={PhD Thesis},   % subject of the document
    pdfcreator={Author Name},   % creator of the document
    pdfproducer={Author Name}, % producer of the document
    pdfkeywords={}, % list of keywords
    pdfnewwindow=true,      % links in new window
    colorlinks=true,       % false: boxed links; true: colored links
    linkcolor=black,          % color of internal links
    citecolor=black,        % color of links to bibliography
    filecolor=black,      % color of file links
    urlcolor=black           % color of external links
}


%\usepackage{svg}
%--------From BRAIN --------%
\usepackage{tabularx}
\RequirePackage[english=usenglishmax]{hyphsubst}
\usepackage{calc}
\usepackage{multirow}

\usepackage{tabularx,ragged2e,booktabs}
\newcolumntype{C}[1]{>{\Centering}m{#1}}
\renewcommand{\arraystretch}{1.3}
%\usepackage{dblfloatfix}

\begin{document}

\frontmatter

{\setstretch{2}
\newcommand{\HRule}{\rule{8cm}{0.5mm}} 

\begin{titlepage}
\begin{center}

%	TITLE
\sffamily
\bfseries
\LARGE{DISSERTATION TITLE} \\
\vspace{-0.5cm}
\HRule \\
\large

\textsc{
\textrm{
DISSERTATION PROPOSAL}}\\

\mdseries
%	DEGREEE
{for the Degree of} \\[-0.25cm]
\textsc{
\textrm{
DOCTOR OF PHILOSOPHY (Degree Title)}} \\[-0.25cm]
at the \\[-0.25cm]
\textsc{
\textrm{
NEW YORK UNIVERSITY \\
\setlength{\parskip}{-1em}
TANDON SCHOOL OF ENGINEERING}} \\[0.5cm]

%	AUTHOR
by \\
\textrm{
\large{
Author Name}} \\[-0.25cm]

%	DATE
\textrm{
\large{
{\today}}}

\end{center}

%%%%%%% SIGNATURE SECTION %%%%%%%
\small
\vspace{1.5cm}
Approved by the Guidance Committee :
\vspace{0.5cm}


%   SIGNATURE BLOCK
\hspace{4.5cm} \hrulefill\

\vspace{-0.5cm}

\hspace{4.5cm} 
\textbf{Advisor's Name}

\vspace{-.5cm}

\hspace{4.5cm} Professional Rank/Title

\vspace{-0.25cm}

\hspace{4.5cm} \hrulefill\

\vspace{-0.5cm}

\hspace{4.5cm}
Date

%   SIGNATURE BLOCK
\hspace{4.5cm} \hrulefill\

\vspace{-0.5cm}

\hspace{4.5cm}
\textbf{Coadvisor's/Committee Member's Name}

\vspace{-.5cm}

\hspace{4.5cm} Professional Rank/Title

\vspace{-0.25cm}

\hspace{4.5cm} \hrulefill\

\vspace{-0.5cm}

\hspace{4.5cm}
Date

%   SIGNATURE BLOCK
\hspace{4.5cm} \hrulefill\

\vspace{-0.5cm}

\hspace{4.5cm}
\textbf{Coadvisor's/Committee Member's Name}

\vspace{-.5cm}

\hspace{4.5cm} Professional Rank/Title

\vspace{-0.25cm}

\hspace{4.5cm} \hrulefill\

\vspace{-0.5cm}

\hspace{4.5cm}
Date

%   SIGNATURE BLOCK
\hspace{4.5cm} \hrulefill\

\vspace{-0.5cm}

\hspace{4.5cm}
\textbf{Coadvisor's/Committee Member's Name}

\vspace{-.5cm}

\hspace{4.5cm} Professional Rank/Title

\vspace{-0.25cm}

\hspace{4.5cm} \hrulefill\

\vspace{-0.5cm}

\hspace{4.5cm}
Date
\end{titlepage}
\setcounter{page}{2}
\include{committee}}

\doublespacing

%\vspace*{-20in}
\onehalfspacing
%\singlespacing
\addcontentsline{toc}{section}{Abstract}
\newcommand{\HRule}{\rule{8cm}{0.5mm}}

\begin{center}

%	TITLE
\sffamily
\LARGE
\textbf{ABSTRACT} \\
\HRule \\
\textsc{
\textrm{DISSERTATION TITLE}}\\[0.25cm]

%	AUTHOR
\normalsize{by} \\[0.5cm]
\Large
\textrm{
% \large
Author Name\\[0.5cm]
Advisor: Prof. Your Advisor, Ph.D.}\\[.5cm]

%	DEGREE INFO
\normalsize
\textrm{Dissertation Proposal for the Degree of \\
Doctor of Philosophy (Degree Title)} \\[0.5cm]

%	DATE
\textrm{
{\today}} \\[1cm]

\end{center}

\normalsize
This is an NYU Tandon dissertation proposal template. It is awesome because ...

\doublespacing

\renewcommand{\contentsname}{Table of Contents}
\tableofcontents
\listoffigures
\listoftables
\listofalgorithms

\afterpreface

\mainmatter

\doublespacing

%   Introduction
\include{Introduction/introduction}

% ------Study One----------
\chapter{Study One}
\label{ch-1}

\section{Research Question}\label{sec:ChapOneIntroduction}
\input{ChapOne/Section-Question}

\section{Background}\label{sec:ChapOneMotivation}
\input{ChapOne/Section-Background.tex}

\section{Research Design and Methods}\label{sec:ChapOneMethod}
\input{ChapOne/Section-Method}

\section{Preliminary Work}\label{sec:ChapOneWork}
\input{ChapOne/Section-Preliminary.tex}

\section{Required Resources}\label{sec:ChapOneDiscussion}
\input{ChapOne/Section-Resources}


%-------Study Two----------
\chapter{Study Two}
\label{ch-2}

\section{Research Question}\label{sec:ChapTwoIntroduction}
\input{ChapTwo/Section-Question}

\section{Background}\label{sec:ChapTwoMotivation}
\input{ChapTwo/Section-Background}

\section{Research Design and Methods}\label{sec:ChapTwoMethod}
\input{ChapTwo/Section-Method}

\section{Preliminary Work}\label{sec:ChapTwoResults}
\input{ChapTwo/Section-Preliminary}

\section{Required Resources}\label{sec:ChapTwoDiscussion}
\input{ChapTwo/Section-Resources.tex}


% ------Study Three--------
\chapter{Study Three}
\label{ch-3}

\section{Research Question}\label{sec:ChapThreeIntroduction}
\input{ChapThree/Section-Question.tex}

\section{Background}\label{sec:ChapThreeMotivation}
\input{ChapThree/Section-Background.tex}

\section{Research Design and Methods}\label{sec:ChapThreeMethod}
\input{ChapThree/Section-Method}

\section{Preliminary Work}\label{sec:ChapThreeResults}
\input{ChapThree/Section-Preliminary.tex}

\section{Required Resources}\label{sec:ChapThreeDiscussion} 
\input{ChapThree/Section-Resources.tex}

%   Conclusion
\include{Conclusion/conclusion}

% Appendices
\nonumchapter{Appendix}

\phantomsection
\section*{Figures}
\addcontentsline{toc}{section}{Figures}

\begin{figure}
\includegraphics[width=8cm]{figures/fig1}
\caption[Short figure name.]{This is a figure that floats inline and here is its caption.
\label{fig:myInlineFigure55}}
\end{figure}

{\singlespace
\printbibliography}

%   Vita
\addcontentsline{toc}{chapter}{Vita}
% \newcommand{\HRule}{\rule{8cm}{0.5mm}}

%   PERSONAL INFO
\begin{center}
\textsc{\textbf{\LARGE{Author Name}}} \\
\HRule \\
Date of Birth: MM/DD/YYYY \\[-0.25cm]
Place of Birth: City, State, Country
\end{center}

%   RESEARCH LOCATION(S) AND DATE(S)
\vspace{2pt}
\begin{center}
\textrm{\textbf{\large{\underline{Research Affiliations}}}}\end{center}

\begin{multicols}{2}
\flushleft{
\small{University Name} \\[-0.25cm]
\textit{Lab Name} 
\vspace{1pt}}
\columnbreak
\flushright{
\small{Date Range} \\
}
\end{multicols}

\vspace{-0.75cm}
\begin{multicols}{2}
\flushleft{
\small{University Name} \\[-0.25cm]
\textit{Lab Name} 
\vspace{1pt}}
\columnbreak
\flushright{
\small{Date Range} \\
}
\end{multicols}

%   RESEARCH SUPPORT
\begin{center}
\textrm{\textbf{\large{\underline{Research Support}}}}\end{center}

\begin{multicols}{2}
\flushleft{
\small{Supporting Institution} \\[-0.25cm]
\textit{Support Designation} 
\vspace{1pt}}
\columnbreak
\flushright{
\small{Date Range} \\
}
\end{multicols}

\vspace{-0.75cm}
\begin{multicols}{2}
\flushleft{
\small{Supporting Institution} \\[-0.25cm]
\textit{Support Designation} 
\vspace{1pt}}
\columnbreak
\flushright{
\small{Date Range} \\
}
\end{multicols}

%   PUBLICATIONS
\vspace{2pt}
\begin{center}\textrm{\textbf{\large{\underline{Publications}}}}\end{center}
% \nocite{einstein}
% \nocite{ctan}
% \nocite{knuthwebsite}
% \nocite{latexcompanion}
% \nocite{knuth-fa}
% \nocite{knuth-acp}
{\singlespace
\vspace{-0.5cm}
\printbibliography[env=nolabelbib,type=article,heading=none]}


\end{document}
